% !TEX root = essential.tex
% !TEX recipe = LuaLaTeX+se
%\documentclass{article}
\documentclass[parskip]{scrartcl} % set document class: manual at https://ctan.org/pkg/koma-script
%\usepackage{fontspec}

% Load packages:
\usepackage[osf,p]{libertine} % set font
\usepackage[autocompile]{gregoriotex} % enable Gregorio score inclusion
\usepackage[latin]{babel} % set language
\usepackage{multicol}
\usepackage[margin=3cm]{geometry}
\usepackage{scrlayer-scrpage}
%\usepackage{musixtex}

\newcommand{\myhome}{C:/Users/johna/Chant/}
\gresetgregpath{{\myhome}}

\usepackage{hyperref}

\hypersetup{colorlinks=true,linkcolor=blue,urlcolor=blue}
\urlstyle{rm}


\setkomafont{section}{\normalfont\centering\Large\scshape} % section heading style
\setkomafont{subsection}{\normalfont\centering\large\bfseries} % section heading style
\setkomafont{subsubsection}{\normalfont\centering\small\scshape} % section heading style
\setcounter{secnumdepth}{-\maxdimen} % remove section numbering

\setkomafont{pagehead}{\scshape}
\automark[section]{section}
%\automark*[subsection]{subsection}

%\lohead[\thesection]
%{\thesection}
%\rohead[\thesubsection]
%{\thesubsection}
\pagestyle{scrheadings}

\def\translation{NABRE}

% Set the space around the initial:
% See http://gregorio-project.github.io/gregoriotex/details.html for more details and options
\grechangedim{beforeinitialshift}{2.2mm}{scalable}
\grechangedim{afterinitialshift}{2.2mm}{scalable}

% Set the initial font (change 43 for a larger size):
%\grechangestyle{initial}{\fontsize{43}{43}\selectfont}%

% Make staff lines red; remove for black:
%\gresetlinecolor{gregoriocolor}

% Use the "commentary" field of the score in the top right corner:
\gresetheadercapture{commentary}{grecommentary}{string}

% Format annotation above initial
\grechangestyle{annotation}{\small\bfseries}

\begin{document}

\title{Essential Chants for Latin Rite Catholics}
%\subtitle{The New World Gradual}
\author{Prepared by Dr. J \& Mr. Bumbarger}
\subtitle{For the St Mark's Schola Cantorum\\
St. Mark the Evangelist Catholic Church \& School\\
Plano, TX}
\date{April 29, 2025}
\maketitle

% \subsection{Foreword}
% This document is intended to be a small compilation of some of the most
% core chants in Catholic culture throughout the ages. The title specifies
% \emph{Latin Rite} Catholics to acknowledge the many other fully Catholic
% rites of the Church that have their own chant traditions.
% This document will certainly be added to over the years, but I did want to
% keep it on the smaller side, and these are what I came up with for now.

\subsection{The sign of the Cross}
\gabcsnippet{
    (c4) In(g) nó(h)mi(h)ne(h) Pat(h)ris,(g) et(h) Fí(i)li(h)i,(h) (,) et(h) Spí(h)ri(g)tus(h) Sanc(hg)ti.(g) (::) <sp>R/</sp>.<b> A(g)men.</b>(gh) (::)
}
Or,
\gabcsnippet{
    (c4)In(g) the(h) name(h) of(h) the(h) Fa(h)ther,(h) and(h) of(g) the(h) Son,(i) (,) and(h) of(h) the(h) Ho(g)ly(h) Spir(hg)it.(g) (::)
    <sp>R/</sp>.<b> A(g)men.(gh) </b> (::)
}

\subsection{The Our Father}
% \paragraph{Tone A}
Everyone knows the one we do at Mass. This is similar, but more solemn.
\gregorioscore{Communes/pater-noster/English-A.gabc}
% \paragraph{Tone C} The \emph{Graduale} recommends this tone on feast days.
% \gregorioscore{Communes/pater-noster/English-C.gabc}
% \paragraph{A simple tone} This is appropriate perhaps when chanting Lauds or Vespers
% on a weekday.
% \gregorioscore{Communes/pater-noster/English-ferialis.gabc}

\subsection{Hail Mary}
\gregorioscore{Varia/ave-maria/solesmes.gabc}
Or, in English:
\gregorioscore{Varia/ave-maria/english.gabc}

\subsection{Glory Be}
Typically, the Glory Be, or \emph{Gloria Patri}, is prayed
after some other prayer or psalm, and, when sung, is sung using
the psalm tone that matches the preceding chant. (A few Psalm tones are
later in this booklet.)

Supposing you were singing the Hail Mary above (which is in Mode 1,)
and wanted to end with a \emph{Gloria Patri},
Introit tone 1 would be appropriate:
\greannotation{1}
\gabcsnippet{
    (c4) Gló(f)ri(gh)a(h) Pa(h)tri,(h) et(h) Fí(h)li(h)o,(h.) (,) et(h) Spi(h)rí(hj)tu(h)i(h) San(hg~)cto.(gh..) *(:) Sic(gf)ut(gh) e(h)rat(h) in(h) prin(h)cí(h)pi(h)o,(h.) (,) et(h) nunc,(hj) et(h) sem(hg~)per,(gh..) (:) et(gf) in(gh) <nlba>saé(h)cu(h)la(h) sae(h)cu(hjh)ló(g')rum.</nlba>(f) A(fff)men.(d.) (::) 
}

\section{Chants for Exposition \& Benediction of the Blessed Sacrament}
When the priest places the Blessed Sacrament in the monstrance and kneels
to incense it, all kneel and sing \emph{O Salutaris Hostia.} There are many
other tunes for this hymn, but here is one.
\gregorioscore{hymns/o-salutaris-hostia/solesmes-ii-tone-7.gabc}

After a period of Adoration, the priest blesses all present with the
Blessed Sacrament; but before this, all sing the Tantum Ergo:
\gregorioscore{hymns/tantum-ergo/solesmes.gabc} 

\section{The Marian Antiphons}
Traditionally sung after \emph{Compline}, or Night Prayer, before bed.
In monasteries where obligatory silence is observed after Compline, this makes
the Marian antiphon the last sound the religious soul utters for the rest of the day.
The Time-after-Pentecost antiphon, \emph{Salve Regina}, is also traditionally
prayed after the Rosary, though nothing I am aware of prevents one from substituting the
antiphon of the season for the same purpose.

\subsection{Alma Redemptoris Mater - Advent thru Feb. 1}
\gregorioscore{hymns/alma-redemptoris/solesmes.gtex}

\paragraph{Translation}
{\itshape
    Loving mother of the Redeemer,\\
    gate of heaven, star of the sea,\\
    assist your people who have fallen yet strive to rise again,\\
    To the wonderment of nature you bore your Creator,\\
    yet remained a virgin after as before,\\
    You who received Gabriel's joyful greeting,\\
    have pity on us poor sinners.}


\subsection{Ave Regina Caelorum - Candlemas (Feb. 2) thru Holy Saturday}
\gregorioscore{hymns/ave-regina-caelorum/solesmes.gtex}

\paragraph{Translation}
\begin{multicols}{2}
    \itshape
    Hail, O Queen of Heaven.\\
Hail, O Lady of Angels\\
Hail! thou root, hail! thou gate\\
From whom unto the world a light has arisen:

Rejoice, O glorious Virgin,\\
Lovely beyond all others,\\
Farewell, most beautiful maiden,\\
And pray for us to Christ.    
\end{multicols}

\subsection{Regina Caeli}
Sung throughout the Easter Season, both after Compline and, traditionally,
in place of the Angelus.
\gregorioscore{hymns/regina-caeli/solesmes.gabc}

\paragraph{Translation}
\emph{Queen of heaven, rejoice, alleluia.\\
The Son you merited to bear, alleluia,\\
Has risen as he said, alleluia.\\
Pray to God for us, alleluia.}

The Regina Caeli may conclude with:
\gabcsnippet{
    (c4) Re(j)joice(j) and(j) be(j) glad,(j) O(j) Vir(j)gin(j) Mar(j)y,(j) Al(j)le(j)lu(j)ia!(h.) (:) <sp>R/</sp><b> For(j) the(j) Lord(j) has(j) tru(j)ly(j) ris(j)en,(j) Al(j)le(j)lu(j)ia!(h.)</b> (:Z)
    Let(j) us(j) pray:(h.) (:)
    O(j) God,(j) who(j) gave(j) joy(j) to(j) the(j) world(j) through(j) the(j) res(j)ur(j)rec(j)tion(j) of(j) Thy(j) Son,(j) our(j) Lord(j) Je(j)sus(h) Christ,(h.) (;) grant(j) we(j) be(j)seech(j) Thee,(j) that(j) through(j) the(j) in(j)ter(j)ces(j)sion(j) of(j) the(j) Vir(j)gin(j) Mar(j)y,(i) His(h) Moth(j)er,(j.) (;) we(j) may(j) ob(j)tain(j) the(j) joys(j) of(j) ev(j)er(j)last(j)ing(f) life.(f.) (:) Through(j) the(j) same(j) Christ(j) our(h) Lord.(h.) (::)
    <sp>R/</sp><b>A(j.)men.</b>(j.) (::)
}

\subsection{Salve Regina}
Sung from Compline on Pentecost Sunday until Advent, as well as after the Rosary.
\gregorioscore{hymns/salve-regina/simplex-liber-usualis.gabc}
The Rosary may conclude:
\gabcsnippet{
    (c3) Pray(j) for(j) us,(j) O(j) Ho(j)ly(j) Moth(j)er(j) of(h) God.(h.) (:)
    <sp>R/</sp><b>That(j) we(j) may(j) be(j) made(j) wor(j)thy(j) of(j) the(j) prom(j)is(j)es(j) of(h) Christ.</b>(h.) (:Z)
    Let(j) us(h) pray.(h.) (:)
    O(j) God,(j) whose(j) on(j)ly(j) be(j)got(j)ten(j) Son,(j) by(j) His(j) life,(j) death,(j) and(j) res(j)ur(j)rec(j)tion,(j) has(j) pur(j)chased(j) for(j) us(j) the(j) re(j)wards(j) of(j) e(j)ter(j)nal(j) life,(h) †(,)
    grant,(j) we(j) be(j)seech(j) Thee,(j) that(j) med(j)i(j)tat(j)ing(j) on(j) these(j) mys(j)ter(j)ies(j) of(j) the(j) most(j) ho(j)ly(j) Ro(j)sa(j)ry(j) of(j) the(j) Bless(j)ed(j) Vir(i)gin(h) Mar(j)y,(j.) (;)
    we(j) may(j) im(j)i(j)tate(j) what(j) they(j) con(j)tain(j) and(j) ob(j)tain(j) what(j) they(j) prom(j)ise.(h.) (:) Through(j) the(j) same(j) Christ(j) our(j) Lord.(h.) (::) <sp>R/</sp><b>A(j.)men.</b>(j.)
}

\section{Ave Maris Stella}
\includegraphics[width=0.99\textwidth]{hymns/ave-maris-stella/oneline.png}

\section{Veni Creator Spiritus}
\includegraphics[width=0.99\textwidth]{hymns/veni-creator/oneline.png}

%\subsection{Aeterne rerum Conditor}
%\includegraphics[width=0.99\textwidth]{hymns/aeterne-rerum-conditor/oneline.png}

\section{The Psalm Tones}
Below are, in order, most of the tones; many have other endings,
but only one is provided here. A more complete list can be obtained
by searching ``Fr Kevin Vogel Psalm tones'' on Youtube; he has a playlist
with recordings of most of the other endings for these tones.

\greannotation{1f.}
\gabcsnippet{(c4)Cry(f) out(gh) with(h) joy(h) to(h) the(h) Lord,(h) all(h) the(h) earth. †(g.) Ser(h)ve(h) the(h) <b>Lord</b>(ixi hr) with(h) <b>glad</b>(g hr)ness.(h.) * (:) Come(h) be(h)fore(h) him,(h) <i>sin</i>(g)<i>ging</i>(f) <b>for</b>(gh gr) joy.(gf..) (::)}

\greannotation{2.}
\gabcsnippet{(f3)Cry(e) out(f) with(h) joy(h) to(h) the(h) Lord,(h) all(h) the(h) earth. †(f.) Ser(h)ve(h) the(h) Lord(h) with(h) <b>glad</b>(i hr)ness.(h.) * (:) Come(h) be(h)fore(h) him,(h) sin(h)<i>ging</i>(g) <b>for</b>(ef fr) joy.(f.) (::)}

\greannotation{3.b}
\gabcsnippet{(c4)Cry(g) out(hi) with(i) joy(i) to(i) the(i) Lord,(i) all(i) the(i) earth. †(h.) Ser(i)ve(i) the(i) <b>Lord</b>(k jr) with(j jr[ocb:1{]) <b>glad</b>(ih[ocb:0}])ness.(j.) * (:) Come(i) be(i)fore(i) him,(i) <b>sin</b>(j hr)ging(h) <b>for</b>(j jr) joy.(i.) (::)}
Alternate (more recent) version of tone 3:
\greannotation{3.b}
\gabcsnippet{(c4)Cry(g) out(hj) with(j) joy(j) to(j) the(j) Lord,(j) all(j) the(j) earth. †(h.) Ser(j)ve(j) the(j) <b>Lord</b>(k jr) with(j jr[ocb:1{]) <b>glad</b>(ih[ocb:0}])ness.(j.) * (:) Come(j) be(j)fore(j) him,(j) sin(j)<i>ging</i>(h) <b>for</b>(j jr) joy.(i.) (::)}

\greannotation{4.E}
\gabcsnippet{
    (c4)Cry(h) out(gh) with(h) joy(h) to(h) the(h) Lord,(h) all(h) the(h) earth. †(g.) Ser(h)ve(h) the(h) <i>Lord</i>(g) <i>with</i>(h) <b>glad</b>(i hr)ness.(h.) * (:) Come(h) be(h)fore(h) <i>him</i>,(g) <i>sing</i>(h)<i>ing</i>(ih gr[ocb:1{]) <b>for</b>(gf[ocb:0}]) joy.(e.) (::)
}

% An alternate tone 4 has the tenor on \emph{re} instead of \emph{la}.
% Depending on the antiphonal, it may be annotated ``4 alt'', ``2*'', or ``4*''.
% \greannotation{4*.d}
% \gabcsnippet{(c3)Cry(i) out(hi) with(i) joy(i) to(i) the(i) Lord,(i) all(i) the(i) earth. †(h.) Ser(i)ve(i) the(i) <i>Lord</i>(h) <i>with</i>(i) <b>glad</b>(j ir)ness.(i.) * (:) Come(i) be(i)fore(i) <i>him</i>,(h) <i>sing</i>(i)<i>ing</i>(j) <b>for</b>(h ir) joy.(i.) (::)
% }

\greannotation{5}
\gabcsnippet{(c3)Cry(d) out(f) with(h) joy(h) to(h) the(h) Lord,(h) all(h) the(h) earth. †(f.) Ser(h)ve(h) the(h) Lord(h) with(h) <b>glad</b>(i hr)ness.(h.) * (:) Come(h) be(h)fore(h) him,(h) <b>sing</b>(i gr)ing(g) <b>for</b>(h fr) joy.(f.) (::)
}

\greannotation{6}
\gabcsnippet{(c4)Cry(f) out(gh) with(h) joy(h) to(h) the(h) Lord,(h) all(h) the(h) earth. †(g.) Ser(h)ve(h) the(h) Lord(h) <i>with</i>(g) <b>glad</b>(h fr)ness.(f.) * (:) Come(h) be(h)fore(h) him,(h) <i>sing</i>(f)<i>ing</i>(gh) <b>for</b>(g fr) joy.(f.) (::)}

\greannotation{7.a}
\gabcsnippet{(c3)Cry(hg) out(hi) with(i) joy(i) to(i) the(i) Lord,(i) all(i) the(i) earth. †(h.) Ser(i)ve(i) the(i) <b>Lord</b>(k jr) with(j) <b>glad</b>(i jr)ness.(j.) * (:) Come(i) be(i)fore(i) him,(i) <b>sing</b>(j ir)ing(i) <b>for</b>(h hr) joy.(gf..) (::)}

\greannotation{8.G}
\gabcsnippet{(c4)Cry(g) out(h) with(j) joy(j) to(j) the(j) Lord,(j) all(j) the(j) earth. †(h.) Ser(j)ve(j) the(j) Lord(j) with(j) <b>glad</b>(k jr)ness.(j.) * (:) Come(j) be(j)fore(j) him,(j) <i>sing</i>(i)<i>ing</i>(j) <b>for</b>(h gr) joy.(g.) (::)}

% \section{Some Mass chants}
% \subsection{Introit for Gaudete Sunday}
% The ``Introit'' contains the first official text of a particular Mass, so
% many Masses were known by the first words of the Introit.
% Hence, this Introit is why the 3rd Sunday of Advent is called ``Gaudete Sunday.''
% \gregorioscore{Introitus/gaudete/solesmes.gabc}

% \subsection{Introit ``Laetare'' for 4th Sunday in Lent}
% Same note as above.
% \gregorioscore{Introitus/laetare-jerusalem/solesmes.gabc}

\section{Resources}
Official English chant settings for common prayers of the Mass: \url{https://icelweb.org/musicfolder/openmusic.html}

Massive online database of Gregorian Chant scores: \url{https://gregobase.selapa.net/scores.php}

Complete chant settings for the Liturgy of the Hours: \url{https://www.antiphonale.net/oco.pdf}

Complete texts for the Liturgy of the Hours: \url{https://www.ibreviary.org/en/}

Official book of plainchant hymns for the Liturgy of the Hours:\\
Latin: \emph{Liber Hymnarius} Solesmes 1983\\
English: \emph{Divine Office Hymnal} USCCB 2019

Official chant book for the Mass (includes all chant settings for the Kyrie, Gloria, Sanctus, Agnus Dei;
tones for the readings \& prayers, as well as all propers - Introit, Responsorial Gradual, Alleluia/Tract, Offertory and Communion chant for each day of the year): \emph{Graduale Romanum}, Solesmes 1974


\end{document}